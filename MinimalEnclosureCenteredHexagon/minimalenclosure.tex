\documentclass[11pt]{amsart}
\usepackage{geometry}                % See geometry.pdf to learn the layout options. There are lots.
\geometry{letterpaper}                   % ... or a4paper or a5paper or ... 
%\geometry{landscape}                % Activate for for rotated page geometry
%\usepackage[parfill]{parskip}    % Activate to begin paragraphs with an empty line rather than an indent
\usepackage{graphicx}
\usepackage{amssymb}
\usepackage{epstopdf}
\DeclareGraphicsRule{.tif}{png}{.png}{`convert #1 `dirname #1`/`basename #1 .tif`.png}

\title{Minimal Square Enclosure of a Centered Hexagon}
\author{Andrew Reiter}
%\date{}                                           % Activate to display a given date or no date

\begin{document}
\maketitle
\section{}
The centered hexagons are well-known and can be generated as shown by 
Wolfram's mathworld and other locations (insert references). I look at the minimal
square required to enclose a centered hexagon.

If we number the points in the hexagon starting with 0 and iterating in an $Ond$ 
manner (reference!), the end point will be the sequence
\[
c_i = \{ 6, 18, 36, 60, 90, 126, \ldots \}_{i=1}
\]

\textbf{conjecture}
Let $\alpha_0 = 2$ and $\alpha_{i+1} = \alpha_{i} - 3$. Then the length of the sides
of the minimal enclosing square lattice is
\[
\frac{c_i}{6} + \alpha_{i}
\]

%\subsection{}



\end{document}  